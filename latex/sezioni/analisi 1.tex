
\subsection{Misure dirette di resistenze}
I valori riportati in tabella (valori in $\Omega$) sono quelli delle misure dirette delle resistenze, prese col multimetro FLUKE 111; il fondo scala \`e di $200mA$ per le correnti e di $600mV$ per le tensioni.

%tabella RES DIRETTE
\begin{tabella}
	\centering
	\begin{center}
\begin{tabulary}{\textwidth}{CCC}
\toprule
Posizione relativa della lente & Diametro fascio a $30 \cm$ & Diametro fascio a $130 \cm$ \\ \midrule
0.400 & 1.187 & 1.000 \\ \midrule
0.450 & 1.193 & 1.105 \\ \midrule
0.460 & 1.171 & 1.172 \\ \midrule
0.475 & 1.168 & 1.190 \\ \midrule
0.500 & 1.186 & 1.254 \\ \midrule
0.550 & 1.190 & 1.367 \\
\bottomrule
\end{tabulary}
\end{center}

	\caption{Misure dirette resistenze}
	\label{tab:01_tab_1.tex}
\end{tabella}


Per stimare gli errori si \`e usata la formula seguente:
%formula diretta
\[ \sigma_R =\sqrt{ \sigma_{\textrm{sist}}^2 + \sigma_{\textrm{stat}}^2}= 0.58 \sqrt{(R\cdot\Delta P)^2 + (n_{\textrm{digit}} \cdot \min(\textrm{FS}))^2}\]
Infatti gli errori legati alla misurazione sono dovuti sia a errori di scala ($ R= k _R \cdot R^{(r)} $), sia a errori casuali connessi al numero di digit. Per chiarezza di notazione, $\sigma^{(r)}$ \`e considerato errore statistico, mentre con $\sigma$ si intende l'errore totale.

Per quanto riguarda le resistenze $R_5$ e $R_6$ in serie, da una misurazione diretta effettuata col multimetro FLUKE 111 risulta che $R_{\textrm{S,sper}}= (402 \pm 2 )\Omega$. Col calcolo teorico, il valore di tale resistenza equivalente risulta invece $R_{\textrm{S, teor}}=(402 \pm 2 )\Omega$, dove per l'errore teorico \`e stato considerato che $R_{\textrm{S,teor}}=k\cdot(R_1^{(r)} + R_2^{(r)})$, infatti la k \`e costante in misurazioni successive, mantenendo il medesimo fondo scala. Con semplice propagazione degli errori risulta che 
\[\sigma_{R_{\textrm{S,teor}}}=\sqrt{(R_1+R_2)^2\cdot\sigma_k^2+\sigma_{R_1^{(r)}}^2+\sigma_{R_2^{(r)}}^2}\]
dove $\sigma_k$ \`e stata ricavata dall'errore percentuale fornito dal costruttore del multimetro e considerando k distribuito uniformemente:
\(\sigma_k=0.58 \cdot Err\%\). 
\`E stata calcolata la correlazione tra le due diverse stime della resistenza, considerando che la loro differenza dovrebbe essere nulla:
\[\Delta R = R_{\textrm{S,teor}}} - \sigma{R_{\textrm{S,sper}}\]
\[\lambda=\frac{\left|\Delta R - 0\right|}{\sigma_{\Delta R}}=0.5\]
con $\Delta R = k\cdot(R_{\textrm{S,sper}}^{(r)}-R_1^{(r)}-R_2^{(r)})$, da cui per propagazione si ricava che 
\[ \sigma_{\Delta R} = 
        \sqrt{ (\Delta R)^2 \sigma_k^2 + 3 \sigma_R^{(r) 2} }\]

Per il calcolo della resistenza equivalente a $R_5$ e $R_6$ in parallelo, il valore misurato con il multimetro FLUKE 111 \`e $R_{\textrm{P,sper}}= (94 \pm 2) \Omega$.
Il valore teorico \`e $R_{\textrm{P,teor}}=(94 \pm 0.5) \Omega$: considerando che $R_{\textrm{P,teor}}=k  \frac{R_5^{(r)} R_6^{(r)}}{R_5^{(r)} + R_6^{(r)}}$ e propagando, riutilizzando la medesima semplificazione sull'errore di scala, si ottiene 
\[\sigma_{\textrm{P,teor}}=\sqrt{\left(\frac{R_5 R_6}{R_5+R_6}\right)^2 \sigma_k^2 + \frac{R_5^4 + R_6^4}{(R_5 + R_6)^4}  \sigma_{R_{\textrm{P,teor}}^{(r)}}^2} .\]
Per il calcolo della compatibilit\`a, si sono utilizzate le medesime formule che per le resistenze in serie, opportunamente adattate (con la stessa convenzione per $\Delta R$):
\[\lambda=\frac{\left|\Delta R - 0\right|}{\sigma_{\Delta R}}=0.24\]
con $\Delta R = k\cdot(R_{\textrm{S,sper}}^{(r)}-R_1^{(r)}-R_2^{(r)})$, da cui per propagazione si ricava che 
\[ \sigma_{\Delta R} = 
        \sqrt{ (\Delta R)^2 \sigma_k^2 + 3 \sigma_R^{(r) 2} }\]


%%\[\lambda=
%%	\frac{ \left|(R_{\textrm{P,teor}}-R_{\textrm{P,sper}})-0 \right| } { \sigma_{R_{\textrm{P,teor}}}-\sigma_{R_{\textrm{P,sper}}}}=0.14.\] 

Nota bene: tutti i calcoli sono stati effettuati mantenendo un numero superiore di cifre significative, riducendone il numero solo in sede di presentazione dati.



 











\subsection{Misura voltamperometrica di una resistenza}
Per misurare una resistenza piccola \`e stato costruito un circuito come in figura (da aggiungersi). Una prima misura diretta \`e stata effettuata utilizzando il multimetro FLUKE 111, che \`e risultata $R_x=(3.0 \pm 0.1) \Omega$.
Costruito il circuito, si \`e variata la resistenza di carico e la potenza erogata dal generatore per indagare di quanto fosse la caduta di potenziale al variare della corrente che attraversa R. I dati ottenuti sono riportati in tabella. 

%tabella AMPERE-VOLT
\begin{tabella}
	\centering
	\begin{center}
\begin{tabulary}{\textwidth}{CC}
\toprule
i (mA) & V (mV) \\ \midrule
25.0 & 70.5 \\ \midrule
30.6 & 86.2 \\ \midrule
37.5 & 106.4 \\ \midrule
49.6 & 140.3 \\ \midrule
60.8 & 171.7 \\ \midrule
69.7 & 182.0 \\ \midrule
72.9 & 204.3 \\ \midrule
81.8 & 230.1 \\ \midrule
100.0 & 280.1 \\ \midrule
90.5 & 254.8 \\ 
\bottomrule
\end{tabulary}
\end{center}

	\caption{Misure caduta di potenziale}
	\label{tab:02_tab_1.tex}
\end{tabella}

In grafico sono riportate tali misure esprimendo V in funzione di I, sovrapposte a un fit lineare ottenuto col metodo della massima verosimiglianza.

\begin{grafico}
\centering
\begin{tikzpicture}[gnuplot]
%% generated with GNUPLOT 4.6p4 (Lua 5.1; terminal rev. 99, script rev. 100)
%% dom 29 mar 2015 11:48:38 CEST
\path (0.000,0.000) rectangle (12.500,8.750);
\gpcolor{color=gp lt color border}
\gpsetlinetype{gp lt border}
\gpsetlinewidth{1.00}
\draw[gp path] (1.504,0.985)--(1.684,0.985);
\draw[gp path] (11.947,0.985)--(11.767,0.985);
\node[gp node right] at (1.320,0.985) { 50};
\draw[gp path] (1.504,2.218)--(1.684,2.218);
\draw[gp path] (11.947,2.218)--(11.767,2.218);
\node[gp node right] at (1.320,2.218) { 100};
\draw[gp path] (1.504,3.450)--(1.684,3.450);
\draw[gp path] (11.947,3.450)--(11.767,3.450);
\node[gp node right] at (1.320,3.450) { 150};
\draw[gp path] (1.504,4.683)--(1.684,4.683);
\draw[gp path] (11.947,4.683)--(11.767,4.683);
\node[gp node right] at (1.320,4.683) { 200};
\draw[gp path] (1.504,5.916)--(1.684,5.916);
\draw[gp path] (11.947,5.916)--(11.767,5.916);
\node[gp node right] at (1.320,5.916) { 250};
\draw[gp path] (1.504,7.148)--(1.684,7.148);
\draw[gp path] (11.947,7.148)--(11.767,7.148);
\node[gp node right] at (1.320,7.148) { 300};
\draw[gp path] (1.504,8.381)--(1.684,8.381);
\draw[gp path] (11.947,8.381)--(11.767,8.381);
\node[gp node right] at (1.320,8.381) { 350};
\draw[gp path] (1.504,0.985)--(1.504,1.165);
\draw[gp path] (1.504,8.381)--(1.504,8.201);
\node[gp node center] at (1.504,0.677) { 20};
\draw[gp path] (3.593,0.985)--(3.593,1.165);
\draw[gp path] (3.593,8.381)--(3.593,8.201);
\node[gp node center] at (3.593,0.677) { 40};
\draw[gp path] (5.681,0.985)--(5.681,1.165);
\draw[gp path] (5.681,8.381)--(5.681,8.201);
\node[gp node center] at (5.681,0.677) { 60};
\draw[gp path] (7.770,0.985)--(7.770,1.165);
\draw[gp path] (7.770,8.381)--(7.770,8.201);
\node[gp node center] at (7.770,0.677) { 80};
\draw[gp path] (9.858,0.985)--(9.858,1.165);
\draw[gp path] (9.858,8.381)--(9.858,8.201);
\node[gp node center] at (9.858,0.677) { 100};
\draw[gp path] (11.947,0.985)--(11.947,1.165);
\draw[gp path] (11.947,8.381)--(11.947,8.201);
\node[gp node center] at (11.947,0.677) { 120};
\draw[gp path] (1.504,8.381)--(1.504,0.985)--(11.947,0.985)--(11.947,8.381)--cycle;
\node[gp node center,rotate=-270] at (0.246,4.683) {V (mV)};
\node[gp node center] at (6.725,0.215) {i (mA)};
\node[gp node right] at (10.479,1.627) {Retta interpolante};
\gpcolor{rgb color={1.000,0.004,0.004}}
\gpsetlinetype{gp lt plot 0}
\draw[gp path] (10.663,1.627)--(11.579,1.627);
\draw[gp path] (1.504,1.149)--(1.609,1.219)--(1.715,1.289)--(1.820,1.359)--(1.926,1.429)%
  --(2.031,1.499)--(2.137,1.568)--(2.242,1.638)--(2.348,1.708)--(2.453,1.778)--(2.559,1.848)%
  --(2.664,1.917)--(2.770,1.987)--(2.875,2.057)--(2.981,2.127)--(3.086,2.197)--(3.192,2.267)%
  --(3.297,2.336)--(3.403,2.406)--(3.508,2.476)--(3.614,2.546)--(3.719,2.616)--(3.825,2.686)%
  --(3.930,2.755)--(4.036,2.825)--(4.141,2.895)--(4.247,2.965)--(4.352,3.035)--(4.458,3.104)%
  --(4.563,3.174)--(4.669,3.244)--(4.774,3.314)--(4.880,3.384)--(4.985,3.454)--(5.090,3.523)%
  --(5.196,3.593)--(5.301,3.663)--(5.407,3.733)--(5.512,3.803)--(5.618,3.873)--(5.723,3.942)%
  --(5.829,4.012)--(5.934,4.082)--(6.040,4.152)--(6.145,4.222)--(6.251,4.291)--(6.356,4.361)%
  --(6.462,4.431)--(6.567,4.501)--(6.673,4.571)--(6.778,4.641)--(6.884,4.710)--(6.989,4.780)%
  --(7.095,4.850)--(7.200,4.920)--(7.306,4.990)--(7.411,5.059)--(7.517,5.129)--(7.622,5.199)%
  --(7.728,5.269)--(7.833,5.339)--(7.939,5.409)--(8.044,5.478)--(8.150,5.548)--(8.255,5.618)%
  --(8.361,5.688)--(8.466,5.758)--(8.571,5.828)--(8.677,5.897)--(8.782,5.967)--(8.888,6.037)%
  --(8.993,6.107)--(9.099,6.177)--(9.204,6.246)--(9.310,6.316)--(9.415,6.386)--(9.521,6.456)%
  --(9.626,6.526)--(9.732,6.596)--(9.837,6.665)--(9.943,6.735)--(10.048,6.805)--(10.154,6.875)%
  --(10.259,6.945)--(10.365,7.015)--(10.470,7.084)--(10.576,7.154)--(10.681,7.224)--(10.787,7.294)%
  --(10.892,7.364)--(10.998,7.433)--(11.103,7.503)--(11.209,7.573)--(11.314,7.643)--(11.420,7.713)%
  --(11.525,7.783)--(11.631,7.852)--(11.736,7.922)--(11.842,7.992)--(11.947,8.062);
\gpcolor{color=gp lt color border}
\node[gp node right] at (10.479,1.319) {Dati};
\gpcolor{rgb color={0.000,0.000,0.000}}
\gpsetpointsize{4.00}
\gppoint{gp mark 2}{(9.858,6.658)}
\gppoint{gp mark 2}{(7.028,4.789)}
\gppoint{gp mark 2}{(5.765,3.985)}
\gppoint{gp mark 2}{(3.332,2.375)}
\gppoint{gp mark 2}{(4.595,3.211)}
\gppoint{gp mark 2}{(8.866,6.034)}
\gppoint{gp mark 2}{(2.611,1.877)}
\gppoint{gp mark 2}{(2.026,1.490)}
\gppoint{gp mark 2}{(7.958,5.425)}
\gppoint{gp mark 2}{(6.172,4.239)}
\gppoint{gp mark 2}{(11.121,1.319)}
\gpcolor{color=gp lt color border}
\gpsetlinetype{gp lt border}
\draw[gp path] (1.504,8.381)--(1.504,0.985)--(11.947,0.985)--(11.947,8.381)--cycle;
%% coordinates of the plot area
\gpdefrectangularnode{gp plot 1}{\pgfpoint{1.504cm}{0.985cm}}{\pgfpoint{11.947cm}{8.381cm}}
\end{tikzpicture}
%% gnuplot variables

\caption{Fit lineare}
\label{fig:fitlin}
\end{grafico}

I coefficienti della retta interpolante $y=mx+q$ sono:
\[m = (2.804 \pm 0.006) \Omega \]
\[q = (0.6 \pm 0.4) mV.\]
%Calcolando la covarianza tra $m$ e $c$ si ottiene 
%\[cov(m, c) = -0.000660084 V^4\] %UDM
Si \`e calcolata la correlazione 
\[\rho(m, q) = \frac{cov(m, q)}{\sigma_m \sigma_q}=-0.89\]
e l'errore a posteriori sulla caduta di tensione \`e di $\sigma_V=0.6mV$.

A seguire il grafico dei residui: si \`e rappresentata la differenza tra il valore di tensione misurato e quello ricavato teoricamente dalla retta interpolante in corrispondenza del suo valore di corrente.

\begin{grafico}
\centering
\begin{tikzpicture}[gnuplot]
%% generated with GNUPLOT 4.6p4 (Lua 5.1; terminal rev. 99, script rev. 100)
%% dom 29 mar 2015 11:48:38 CEST
\path (0.000,0.000) rectangle (12.500,8.750);
\gpcolor{color=gp lt color border}
\gpsetlinetype{gp lt border}
\gpsetlinewidth{1.00}
\draw[gp path] (1.504,0.985)--(1.684,0.985);
\draw[gp path] (11.947,0.985)--(11.767,0.985);
\node[gp node right] at (1.320,0.985) {-1.5};
\draw[gp path] (1.504,2.218)--(1.684,2.218);
\draw[gp path] (11.947,2.218)--(11.767,2.218);
\node[gp node right] at (1.320,2.218) {-1};
\draw[gp path] (1.504,3.450)--(1.684,3.450);
\draw[gp path] (11.947,3.450)--(11.767,3.450);
\node[gp node right] at (1.320,3.450) {-0.5};
\draw[gp path] (1.504,4.683)--(1.684,4.683);
\draw[gp path] (11.947,4.683)--(11.767,4.683);
\node[gp node right] at (1.320,4.683) { 0};
\draw[gp path] (1.504,5.916)--(1.684,5.916);
\draw[gp path] (11.947,5.916)--(11.767,5.916);
\node[gp node right] at (1.320,5.916) { 0.5};
\draw[gp path] (1.504,7.148)--(1.684,7.148);
\draw[gp path] (11.947,7.148)--(11.767,7.148);
\node[gp node right] at (1.320,7.148) { 1};
\draw[gp path] (1.504,8.381)--(1.684,8.381);
\draw[gp path] (11.947,8.381)--(11.767,8.381);
\node[gp node right] at (1.320,8.381) { 1.5};
\draw[gp path] (1.504,0.985)--(1.504,1.165);
\draw[gp path] (1.504,8.381)--(1.504,8.201);
\node[gp node center] at (1.504,0.677) { 20};
\draw[gp path] (3.593,0.985)--(3.593,1.165);
\draw[gp path] (3.593,8.381)--(3.593,8.201);
\node[gp node center] at (3.593,0.677) { 40};
\draw[gp path] (5.681,0.985)--(5.681,1.165);
\draw[gp path] (5.681,8.381)--(5.681,8.201);
\node[gp node center] at (5.681,0.677) { 60};
\draw[gp path] (7.770,0.985)--(7.770,1.165);
\draw[gp path] (7.770,8.381)--(7.770,8.201);
\node[gp node center] at (7.770,0.677) { 80};
\draw[gp path] (9.858,0.985)--(9.858,1.165);
\draw[gp path] (9.858,8.381)--(9.858,8.201);
\node[gp node center] at (9.858,0.677) { 100};
\draw[gp path] (11.947,0.985)--(11.947,1.165);
\draw[gp path] (11.947,8.381)--(11.947,8.201);
\node[gp node center] at (11.947,0.677) { 120};
\draw[gp path] (1.504,8.381)--(1.504,0.985)--(11.947,0.985)--(11.947,8.381)--cycle;
\node[gp node center,rotate=-270] at (0.246,4.683) {V (mV)};
\node[gp node center] at (6.725,0.215) {i (mA)};
\node[gp node right] at (10.479,8.047) {Residui};
\gpcolor{rgb color={1.000,0.004,0.004}}
\gpsetlinetype{gp lt plot 0}
\draw[gp path] (10.663,8.047)--(11.579,8.047);
\draw[gp path] (10.663,8.137)--(10.663,7.957);
\draw[gp path] (11.579,8.137)--(11.579,7.957);
\draw[gp path] (9.858,1.537)--(9.858,4.317);
\draw[gp path] (9.768,1.537)--(9.948,1.537);
\draw[gp path] (9.768,4.317)--(9.948,4.317);
\draw[gp path] (7.028,1.531)--(7.028,4.311);
\draw[gp path] (6.938,1.531)--(7.118,1.531);
\draw[gp path] (6.938,4.311)--(7.118,4.311);
\draw[gp path] (5.765,4.596)--(5.765,7.375);
\draw[gp path] (5.675,4.596)--(5.855,4.596);
\draw[gp path] (5.675,7.375)--(5.855,7.375);
\draw[gp path] (4.783,4.328)--(4.783,7.108);
\draw[gp path] (4.693,4.328)--(4.873,4.328);
\draw[gp path] (4.693,7.108)--(4.873,7.108);
\draw[gp path] (3.332,4.273)--(3.332,7.053);
\draw[gp path] (3.242,4.273)--(3.422,4.273);
\draw[gp path] (3.242,7.053)--(3.422,7.053);
\draw[gp path] (4.595,4.413)--(4.595,7.193);
\draw[gp path] (4.505,4.413)--(4.685,4.413);
\draw[gp path] (4.505,7.193)--(4.685,7.193);
\draw[gp path] (8.866,4.671)--(8.866,7.450);
\draw[gp path] (8.776,4.671)--(8.956,4.671);
\draw[gp path] (8.776,7.450)--(8.956,7.450);
\draw[gp path] (2.611,2.052)--(2.611,4.832);
\draw[gp path] (2.521,2.052)--(2.701,2.052);
\draw[gp path] (2.521,4.832)--(2.701,4.832);
\draw[gp path] (2.026,1.961)--(2.026,4.740);
\draw[gp path] (1.936,1.961)--(2.116,1.961);
\draw[gp path] (1.936,4.740)--(2.116,4.740);
\draw[gp path] (7.958,3.767)--(7.958,6.547);
\draw[gp path] (7.868,3.767)--(8.048,3.767);
\draw[gp path] (7.868,6.547)--(8.048,6.547);
\draw[gp path] (6.172,3.097)--(6.172,5.876);
\draw[gp path] (6.082,3.097)--(6.262,3.097);
\draw[gp path] (6.082,5.876)--(6.262,5.876);
\gpsetpointsize{4.00}
\gppoint{gp mark 1}{(9.858,2.927)}
\gppoint{gp mark 1}{(7.028,2.921)}
\gppoint{gp mark 1}{(5.765,5.986)}
\gppoint{gp mark 1}{(4.783,5.718)}
\gppoint{gp mark 1}{(3.332,5.663)}
\gppoint{gp mark 1}{(4.595,5.803)}
\gppoint{gp mark 1}{(8.866,6.060)}
\gppoint{gp mark 1}{(2.611,3.442)}
\gppoint{gp mark 1}{(2.026,3.351)}
\gppoint{gp mark 1}{(7.958,5.157)}
\gppoint{gp mark 1}{(6.172,4.486)}
\gppoint{gp mark 1}{(11.121,8.047)}
\gpcolor{rgb color={0.000,0.000,0.000}}
\gpsetlinetype{gp lt plot 1}
\draw[gp path] (1.504,4.683)--(1.609,4.683)--(1.715,4.683)--(1.820,4.683)--(1.926,4.683)%
  --(2.031,4.683)--(2.137,4.683)--(2.242,4.683)--(2.348,4.683)--(2.453,4.683)--(2.559,4.683)%
  --(2.664,4.683)--(2.770,4.683)--(2.875,4.683)--(2.981,4.683)--(3.086,4.683)--(3.192,4.683)%
  --(3.297,4.683)--(3.403,4.683)--(3.508,4.683)--(3.614,4.683)--(3.719,4.683)--(3.825,4.683)%
  --(3.930,4.683)--(4.036,4.683)--(4.141,4.683)--(4.247,4.683)--(4.352,4.683)--(4.458,4.683)%
  --(4.563,4.683)--(4.669,4.683)--(4.774,4.683)--(4.880,4.683)--(4.985,4.683)--(5.090,4.683)%
  --(5.196,4.683)--(5.301,4.683)--(5.407,4.683)--(5.512,4.683)--(5.618,4.683)--(5.723,4.683)%
  --(5.829,4.683)--(5.934,4.683)--(6.040,4.683)--(6.145,4.683)--(6.251,4.683)--(6.356,4.683)%
  --(6.462,4.683)--(6.567,4.683)--(6.673,4.683)--(6.778,4.683)--(6.884,4.683)--(6.989,4.683)%
  --(7.095,4.683)--(7.200,4.683)--(7.306,4.683)--(7.411,4.683)--(7.517,4.683)--(7.622,4.683)%
  --(7.728,4.683)--(7.833,4.683)--(7.939,4.683)--(8.044,4.683)--(8.150,4.683)--(8.255,4.683)%
  --(8.361,4.683)--(8.466,4.683)--(8.571,4.683)--(8.677,4.683)--(8.782,4.683)--(8.888,4.683)%
  --(8.993,4.683)--(9.099,4.683)--(9.204,4.683)--(9.310,4.683)--(9.415,4.683)--(9.521,4.683)%
  --(9.626,4.683)--(9.732,4.683)--(9.837,4.683)--(9.943,4.683)--(10.048,4.683)--(10.154,4.683)%
  --(10.259,4.683)--(10.365,4.683)--(10.470,4.683)--(10.576,4.683)--(10.681,4.683)--(10.787,4.683)%
  --(10.892,4.683)--(10.998,4.683)--(11.103,4.683)--(11.209,4.683)--(11.314,4.683)--(11.420,4.683)%
  --(11.525,4.683)--(11.631,4.683)--(11.736,4.683)--(11.842,4.683)--(11.947,4.683);
\gpcolor{color=gp lt color border}
\gpsetlinetype{gp lt border}
\draw[gp path] (1.504,8.381)--(1.504,0.985)--(11.947,0.985)--(11.947,8.381)--cycle;
%% coordinates of the plot area
\gpdefrectangularnode{gp plot 1}{\pgfpoint{1.504cm}{0.985cm}}{\pgfpoint{11.947cm}{8.381cm}}
\end{tikzpicture}
%% gnuplot variables

\caption{Residui}
\label{fig:residui}
\end{grafico}

Una stima della resistenza \`e data dalla pendenza della retta interpolante. Tale retta ha un errore che \`e composizione di un errore sistematico e di uno statistico, infatti si pu\`o scrivere $m=\frac{k_V (V_2^{(r)}-V_1^{(r)})}{k_i (i_2^{(r)} - i_1^{(r)})}=\frac{k_V}{k_i}m^{(r)}$.
Da una propagazione risulta che l'errore su tale grandezza \`e $\sigma_m=\sqrt{\sigma_{\textrm{m,fit}}^2 + \sigma_{k_V}^2 m^2 + \sigma_{k_i}^2 m^2}$ con $\sigma_{\textrm{m,fit}}$ errore casuale ottenuto dall'interpolazione.
Risulta che l'incertezza sulla resistenza \`e quasi completamente data dall'errore sistematico. Il risultato finale \`e $R=(2.8 \pm 0.4) \Omega$; l'errore percentuale \`e del $13 \%$.

Si possono confrontare il risultato teorico e quello sperimentale con un calcolo di compatibilit\`a. Dato che sono state usate strumentazioni differenti per le due stime, se ne pu\`o applicare la definizione: 
$\lambda=\frac{|R_x - R|}{\sqrt{\sigma_{R_x}^2+\sigma_R^2}}=0.5$.
 











\subsection{Resistenze interne degli strumenti di misura}
Attraverso costruzioni di circuiti o misure dirette, si sono stimate le resistenze interne degli strumenti utilizzati.
Per la stima della resistenza interna del generatore si \`e costruito un circuito come in figura (da aggiungersi) e utilizzato il voltmetro AGILENT U1232A con l'amperometro BECKMAN T110B.
Dalle misure risulta che
\begin{align}
V_0 &=(5.01 \pm 0.01 )V \ \textrm{con}\  V_{\textrm{FS}}=6V \\
i   &=(124.9 \pm 0.5) mA \ \textrm{con}\  i_{\textrm{FS}}=200mA \\
V   &=(5.00 \pm 0.01) V \ \textrm{con}\  V_{\textrm{FS}}=6V.
\end{align}
Da uno studio del circuito si ricava la formula $R_G=\frac{V_0-V}{V}$.
Stimandone l'errore, per evitare problemi di correlazione si pu\`o scrivere $R_G=\frac{k_v (V_0^{(r)}- V^{(r)})}{i}$, da cui propagando: 
\[\sigma_{R_G}=\sqrt{R_G^2 \sigma_{k_V}^2 + \frac{(\sigma_{V^{(r)}}^2 + \sigma_{V_0^{(r)}}^2)}{i^2} + \frac{(V_0-V)^2}{i^4} \sigma_i^2},\] ricordando che per $\sigma_i$ si intende l'errore strumentale totale. Concludendo, $R_G= (0.10 \pm 0.16) \Omega$.

Un diverso circuito \`e stato costruito per stimare la resistenza interna dell'AGILENT U1232A utilizzato come voltmetro.
Una misurazione diretta di $R_V$ \`e stata ottenuta utilizzando come ohmetro il BECKMAN T110B: $R_{\textrm{V, sper}}=(11.2 \pm 0.1) M\Omega$, con fondo scala di $20 M\Omega$. 
Le misure prese a circuito chiuso sono: 
\begin{align}
R_S &= (0.990 \pm 0.005) M\Omega \ \textrm{con}\  R_{\textrm{FS}}=6 M\Omega \\
V_0 &= (5.01 \pm 0.01) V \ \textrm{con}\  V_{\textrm{FS}} = 6 V \\
V &= (4.60 \pm 0.01) V \ \textrm{con}\  V_{\textrm{FS}} = 6 V
\end{align}
Studiando il circuito, si pu\`o dimostrare che 
\[ R_{\textrm{V, teor}} = \frac{R_S V}{V_0 - V} . \]
Portando fuori dai valori il coefficiente $k_V$ e semplificandolo, si ha $R_{\textrm{V, teor}} = \frac{R_S V^{(r)}}{V_0^{(r)} - V^{(r)}}$ da cui, propagando, si ottiene \[\sigma_{R_{\textrm{V,teor}}} = \sqrt{\sigma_{R_S}^2 \left(\frac{V}{(V_0 - V)^2} \right)^2 + \sigma_{V_{(r)}}^2 \left(\frac{R_S V_0}{(V_0 - V)^2}\right)^2 + \sigma_{V_0^{(r) 2}} \left(\frac{R_S V}{(V_0 - V)^2}\right)^2}.\]
Risulta $R_{\textrm{V, teor}} = (11.19 \pm 0.08) M\Omega$.


Per misurare la resistenza interna del BECKMAN T110B, usato come amperometro, si \`e semplicemente effettuato un collegamento con il FLUKE 111 usato come ohmetro. I valori sono riportati in tabella.

%tabella RESISTENZE AMPEROMETRO
\begin{tabella}
	\centering
	\begin{center}
\begin{tabulary}{\textwidth}{CCCC}
\toprule
$I_{FS}$ & $R (\Omega)$ & $\sigma_R (\Omega)$ & $R_{FS} (\Omega)$ \\ \midrule
200 mA & 1002 & 5 & 6000 \\ \midrule
2 mA & 102.1 & 0.5 & 600 \\ \midrule
20 mA & 11.4 & 0.1 & 600 \\ \midrule
200 mA & 1.8 & 0.1 & 600 \\ \midrule
2 A & 1.2 & 0.1 & 600 \\ 
\bottomrule
\end{tabulary}
\end{center}

	\caption{Resistenze dell'amperometro BECKMAN}
	\label{tab:03_tab_1.tex}
\end{tabella}

