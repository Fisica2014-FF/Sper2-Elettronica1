\'E presentata qua la parte fondamentale del codice in c++ usato per i calcoli numerici. Inoltre è stato usato per i calcoli Mathematica.

\begin{lstlisting}[language=C++]
double interp_delta(int N, double * x, double * y)
{
	double delta=N*sumquad(N, x)-((sum(N, x))*(sum(N, x)));
	return delta;
}
double interp_a(int N, double * x, double * y)
{
	double a=(1/interp_delta(N, x, y))*((sumquad(N, x))*(sum(N, y))-(sum(N, x))*(sumprod (N, x, y)));
	return a;
}
double interp_b(int N, double * x, double * y)
{
	double b=(1/interp_delta(N, x, y))*(N*(sumprod(N, x, y))-((sum(N, x))*(sum(N, y))));
	return b;
}
double sigmay_post(int N, double * x, double * y)
{
	double sy=0;
	for (int i=0; i<N; i++){
		sy+=((interp_a(N, x, y))+((interp_b(N, x, y))*x[i])-(y[i]))*((interp_a(N, x, y))+((interp_b(N, x, y))*x[i])-(y[i]));
	}
	sy=sy/(N-2);
	sy=sqrt(sy);
	return sy;
}
double interp_sigmaa(int N, double * x, double * y)
{
	double sa=(sigmay_post(N, x, y))*sqrt((sumquad(N, x))/(interp_delta(N, x, y)));
	return sa;
}
double interp_sigmab(int N, double * x, double * y)
{
	double sb=(sigmay_post(N, x, y))*sqrt(N/(interp_delta(N, x, y)));
	return sb;
}
double cov_mq(int N, double * x, double * y)
{
	return (-(sum(N, x)*sigmay_post(N, x, y)*sigmay_post(N, x, y)/interp_delta(N, x, y)));
}
\end{lstlisting}
