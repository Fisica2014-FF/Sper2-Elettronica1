Un ulteriore metodo per la misurazione del fuoco della lente è quello che prevede l'utilizzo della legge dei punti coniugati.
 A partire dall'equazione di Gauss per le lenti sottili, apportando alcune correzioni legate al modello delle lenti
 spesse, si può trovare una serie di equazioni che descrivono il comportamento delle lenti attraverso semplici misure di
 lunghezze. Tali equazioni sono:
\[p=P_L - P_O + \frac {dr} {2} - \frac {\overline{PP'}}{2}\]
\[q=P_S - P_L -\frac {dr} {2} + \frac {\overline{PP'}} {2}\]
\begin{equation} \label{eq:punticoniugati}
\frac{1}{f} = \frac {1}{p} + \frac {1}{q}
\end{equation}

% con  p\textsubscript{o} distanza oggetto, p\textsubscript{L} distanza lente e p\textsubscript{S} distanza schermo.
 Sono stati prelevati due campioni diversi, dato che il primo campione possedeva delle imprecisioni. I campioni si possono trovare
 nella \autoref{tab:02tab1}.
%**02_tab_1.fdat**	**02_tab_2.txt**
\begin{tabella}
	\centering
	\begin{center}
\begin{tabulary}{\textwidth}{CC}
\toprule
i (mA) & V (mV) \\ \midrule
25.0 & 70.5 \\ \midrule
30.6 & 86.2 \\ \midrule
37.5 & 106.4 \\ \midrule
49.6 & 140.3 \\ \midrule
60.8 & 171.7 \\ \midrule
69.7 & 182.0 \\ \midrule
72.9 & 204.3 \\ \midrule
81.8 & 230.1 \\ \midrule
100.0 & 280.1 \\ \midrule
90.5 & 254.8 \\ 
\bottomrule
\end{tabulary}
\end{center}

	\caption{Campioni (udm in $[\cm\,]$)}
	\label{tab:02tab1}
\end{tabella}
%\begin{tabella}
	%\centering
	%\input{./tabelle/02_tab_2.tex}
	%\caption{Campione II $[\cm\,]$}
	%\label{tab:02tab2}
%\end{tabella}
%
Il valore mantenuto costante durante tutta l'esperienza \`e
\[P_O = (13.15 \pm 0.05) \cm\] 
Tali campioni sono stati presi in momenti diversi, in particolare sono stati riposizionati tutti i cavalieri tra un campione e l'altro
. Si \`e fatto un grafico per cercare di comprendere come i dati si ditribuissero in un piano cartesiano: nel  \autoref{fig:02graph1.tex} è rappresentato $\usuq$ in funzione di $\usup$; secondo la formula
 \eqref{eq:punticoniugati} risulta che il grafico dovrebbe essere una retta con
 pendenza $-1$.
\begin{grafico} \centering \input{../gnuplot/immagini/02_graph_1.tex} \caption{Punti coniugati} \label{fig:02graph1.tex} \end{grafico}

Si evince facilmente come nel "Campione I" ci siano dei punti visibilmente fuori scala, probabilmente a causa di
 inesattezze da parte degli sperimentatori nel prendere quei dati (quando la lente era particolarmente vicina all'oggetto risultava
 meno semplice andare ad identificare in maniera precisa a quale distanza avrebbe dovuto essere lo schermo affinché l'immagine
 fosse a fuoco). Alcune accortezze, prese proprio in conseguenza delle imprecisioni del primo campione, hanno permesso di migliorare
 le misure per un nuovo campione che fosse più preciso del precedente. Ma i due campioni appartengono alla stessa popolazione? La risposta
 risulta negativa: infatti anche nella zona in cui il "Campione I" non è evidentemente sbagliato le due rette non coincidono,
 come si può vedere dal \autoref{fig:02graph2.tex} in cui sono stati riportati i dati non evidentemente sbagliati assieme alla rette
 che li
 interpolano $y = ax + b$, che si possono trovare nella \autoref{tab:02tab3}.
%**02_graph_2.txt**	**02_tab_3.txt**
\begin{grafico} \centering \input{../gnuplot/immagini/02_graph_2.tex} \caption{Le interpolazioni dei due campioni} \label{fig:02graph2.tex} \end{grafico}
\begin{tabella}
	\centering
	\begin{tabulary}{\textwidth}{CCCCC}
\toprule
Campione & $a$  & $\sigma_a$ & $b [\cm^{-1}]$ & $\sigma_b [\cm^{-1}]$\\ \midrule
I & -0.88 & 0.02 & 0.1420 & 0.0006\\ %\midrule
II & -0.986 & 0.004 & 0.1505 & 0.0003\\
\bottomrule
\end{tabulary}


	\caption{Rette interpolanti}
	\label{tab:02tab3}
\end{tabella}

Come è stato già detto, la teoria dice che la retta che rappresentiamo sul piano cartesiano ha pendenza $-1$. Il nostro approccio
 sperimentale, però, inserisce errori di tipo statistico e sistematico che fanno in modo che la retta che effettivamente si trova
 non abbia coefficiente angolare esattamente $-1$, ma ci si avvicini. A questo punto sorge spontaneo chiedersi se convenga di più fare
 un'analisi dati basata su un'interpolazione a doppio parametro (pendenza, intercetta), o fissare la pendenza a $-1$ come vuole la
 teoria andando a cercare quale sia l'intercetta in questo caso. Le due diverse interpolazioni danno origine al
 \autoref{fig:02_graph_3.tex}.
%**02_graph_3.txt**	**02_tab_4.txt**
\begin{grafico} \centering \input{../gnuplot/immagini/02_graph_3.tex} \caption{Le due diverse interpolazioni} \label{fig:02_graph_3.tex} \end{grafico}
\begin{tabella}
	\centering
	\begin{tabulary}{\textwidth}{CCCCC}
\toprule
Parametri del modello & $a$  & $\sigma_a$ & $b [\cm^{-1}]$ & $\sigma_b [\cm^{-1}]$\\ \midrule
$b$ & -1 & - & 0.1514 & 0.0002\\
$a,b$ & -0.986 & 0.004 & 0.1505 & 0.0003\\
\bottomrule
\end{tabulary}


	\caption{Numero parametri d'interpolazione}
	\label{tab:02tab4}
\end{tabella}

Per comprendere quale dei due modi di interpretare il fenomeno sia più opportuno usare si \`e utilizzato il metodo dell'$\mathcal{F}$-test:
 sono stati presi gli scarti quadratici medi delle due interpolazioni diverse e sono stati normalizzati ai gradi di libertà
 (sono state raccolte 16 coppie di dati, ciò vuol dire che la retta con un solo parametro libero ha 15 gradi di libertà,
 siano $n_1$ nelle formule, e la retta con due parametri liberi ha 14 gradi di libertà, $n_2$).
 La formula utilizzata è:
\[\mathcal{F}=\frac{\dfrac{{\sum_i \big( y_i-f_1 (c_i) \big) ^2} - {\sum_i \big( y_i-f_2 (c_i) \big) ^2}}{n_1-n_2}}{\dfrac{\sum_i \big( y_i-f_2 
(c_i) \big) ^2}{n_2}}.\]

Da questa formula risulta $\mathcal{F} = 0.1740$. Ora, per capire quale delle due interpolazioni meglio si addica alla
 casistica trovata, bisogna consultare le tabelle di riferimento relative alla funzione di Von Mises-Fisher, andando a leggere il valore
 in corrispondenza dei gradi di liberà del numeratore (in questo caso 1) per trovare la colonna e i gradi di libertà del denominatore
 (in questo caso $14$) per trovare la riga. Si legge che vale la pena introdurre un nuovo parametro alla teoria 
 se $\mathcal{F} > 3.10$, 
 da cui si evince come non sia necessario introdurre il secondo parametro d'interpolazione della retta, che 
 può essere fissato a $-1$ senza peggiorare il modello\footnote{Si prenda una significanza del $90\%$, 
 riferimento tabella: \url{http://www.socr.ucla.edu/applets.dir/f_table.html}}. 
 Per fare uno studio pi\`u completo si sono effettuate le analisi sia considerando la pendenza della retta come fissata,
 sia considerandola come una variabile casuale dipendente dal campione.

Considerando la pendenza come dipendente dal campione, si trova una retta che interseca gli assi in due punti di coordinate diverse,
 coordinate che indicano proprio due stime diverse del reciproco del fuoco della lente, che chiameremo rispettivamente
$f_x$ se l'intersezione è con l'asse delle $x$ e $f_y$ se l'intersezione è con l'asse delle $y$.
 Per quanto riguarda $f_y$, la sua stima è piuttosto semplice: prendendo l'equazione della retta interpolante $y=ax + b$
 con $a$ e $b$ letti dalla \autoref{tab:02tab4}, basta cercare il punto con la $x$ nulla per trovare l'intersezione con l'asse.
 Con una banale sostituzione si ottiene $f_y= b$ il che dà immediatamente un valore del reciproco del fuoco che sia anche
 accompagnato dalla propria incertezza: con una semplice propagazione si trova che l'incertezza su $f_y$ è uguale a quella su $b$,
 stimata a partire dalle formule di interpolazione che massimizzano la verosimiglianza (anche questo riportato nella 
\autoref{tab:02tab4}).
 Il primo valore del reciproco del fuoco quindi risulta:
\[f_y = (0.1505 \pm 0.0003) \cm^{-1}.\]

Per quanto riguarda l'intersezione con l'asse delle $x$, stavolta il calcolo è leggermente più complesso, infatti sostituendo $y = 0$
 alla formula della retta interpolata si ottiene $f_x = -\frac{q}{m}$.  Per trovare l'incertezza a questo punto risulta necessario
 propagare gli errori su $m$ e su $q$. Applicando la formula di propagazione risulta:
\[\sigma_{f_x} = \sqrt {\frac{1} {m^2} \sigma_q^2 + \frac{q}{m^2} \sigma_m^2 + 2 \cov (q,m) }.\] 
 Sostituendo, il reciproco del fuoco ha un valore di:
\[f_x = (0.153 \pm 0.002) \cm^{-1}.\]

Per trovare il fuoco è necessario calcolare i reciproci di $f_x$ ed $f_y$; secondo la formula di 
propagazione: $f'_x = \frac{1}{f_x}$ e $\sigma_{f'x} = \frac{\sigma_{fx}}{f_{x}^2}$ ed analogo per $f_y$. Da cui i risultati:
\[f'_x = (6.55 \pm 0.09) \cm \]
\[f'_y = (6.64 \pm 0.01) \cm .\]
%
Per dare un risultato finale, si tiene conto di tutti gli errori considerando il fatto che questi due numeri sono diversi: si può 
 stimare effettivamente il fuoco $f$ facendo una media aritmetica dei valori ottenuti e si può stimare l'incertezza
 andando a vedere la
 semidifferenza tra i due valori casuali ottenuti. Il risultato finale è:
\[f = (6.60 \pm 0.04) \cm .\]

Si veda ora come si sarebbero potuti analizzare i dati considerando la retta con l'unico parametro libero l'intercetta. In questo
 caso il risultato del fuoco e un'indicazione della sua incertezza statistica ci vengono dati dai parametri interpolati e
 riportati nella \autoref{tab:02tab4}, per cui la distanza focale, calcolata come nel caso precedente trovando il reciproco
 dell'intercetta e propagando gli errori, sembrerebbe valere:
\[f' = (6.604 \pm 0.009) \cm .\]
Ricordando però che i campioni raccolti per la stima della lunghezza focale attraverso il metodo dei punti coniugati erano due ed
 erano visibilmente poco compatibili tra loro, sorge spontaneo chiedersi quali siano gli errori sistematici collegati a tale metodo.
 Dalle formule della legge \eqref{eq:punticoniugati} risulta che p dipende dalla posizione iniziale dell'oggetto, che tra
 l'altro rimane fisso per tutta la durata dell'esperimento (probabilmente è da ricondurre ad una non perfetta lettura della posizione
 dell'oggetto il fatto che i due campioni non siano sottoinsiemi della stessa popolazione), il che vuol dire che un'errata lettura
 nella posizione iniziale dell'oggetto può sistematicamente creare un bias nel campione, spostando la retta interpolante dalla
 posizione che realmente dovrebbe occupare. Per stimare tale errore sistematico sono stati creati dei campioni fittizi di $\usup$
 considerando l'oggetto non nella posizione registrata al momento dell'esperimento, ma spostata di una sigma (0.5 mm) da
 uno dei due lati. I valori ottenuti sono riassunti nella \autoref{tab:02tab5}.
%**02_tab_5.txt**
\begin{tabella}
	\centering
	\begin{tabulary}{\textwidth}{CCCC}
\toprule
$\usup$ | $P_O$ aumentato & $\usup$ campione & $\usup$ | $P_O$ diminuito & $\usuq$\\ \midrule
0.0596 & 0.0594 & 0.0592 & 0.0922\\ \midrule
0.0634 & 0.0632 & 0.0630 & 0.0869\\ \midrule
0.0677 & 0.0674 & 0.0672 & 0.0844\\ \midrule
0.0726 & 0.0723 & 0.0720 & 0.0803\\ \midrule
0.0782 & 0.0779 & 0.0776 & 0.0733\\ \midrule
0.0849 & 0.0845 & 0.0842 & 0.0671\\ \midrule
0.0927 & 0.0923 & 0.0919 & 0.0593\\ \midrule
0.1022 & 0.1017 & 0.1012 & 0.0499\\ \midrule
0.1139 & 0.1132 & 0.1126 & 0.0396\\ \midrule
0.1285 & 0.1277 & 0.1269 & 0.0244\\ \midrule
0.0459 & 0.0458 & 0.0457 & 0.1058\\ \midrule
0.0373 & 0.0373 & 0.0372 & 0.1130\\ \midrule
0.0272 & 0.0272 & 0.0271 & 0.1242\\ \midrule
0.0205 & 0.0205 & 0.0205 & 0.1307\\ \midrule
0.0162 & 0.0162 & 0.0162 & 0.1342\\ \midrule
0.0130 & 0.0130 & 0.0130 & 0.1379\\
\bottomrule
\end{tabulary}

	\caption{Campioni con errori sistematici $[\cm^{-1}]$}
	\label{tab:02tab5}
\end{tabella}

Rappresentando sia il campione reale sia i campioni fittizi con le relative rette interpolate si vede il 
\autoref{fig:02_graph_4.tex}.
%**02_graf_4.txt**	**02_tab_6.txt**
\begin{grafico} \centering \input{../gnuplot/immagini/02_graph_4.tex} \caption{Errori su $P_O$} \label{fig:02_graph_4.tex} \end{grafico}
\begin{tabella}
	\centering
	\begin{tabulary}{\textwidth}{LCC}
\toprule
Campione & Intercetta $[\cm^{-1}]$  & $\sigma [\cm^{-1}]$\\ \midrule
Campione originario & 0.1514 & 0.0002\\
Con $p_{o}'=(p_o+0.5)\mm$ & 0.1517 & 0.0008\\
Con $p_{o}'=(p_o-0.5)\mm$ & 0.1512 & 0.0001\\
\bottomrule
\end{tabulary}

	\caption{Rette interpolanti errori sistematici}
	\label{tab:02tab6}
\end{tabella}
Da un confronto delle rette interpolanti nella \autoref{tab:02tab6} si può stimare l'incertezza che
 può essere collegata all'imprecisione nella lettura della
 posizione del cavaliere portalampada. In particolare si può leggere l'effettivo valore del fuoco come quello collegato al
 campione realmente raccolto, e ad esso si può associare un errore sistematico legato alla massima distanza tra i fuochi fittizi
 ricavati dalle interpolazioni dei campioni fittizi, che risulta di $0.01\cm$. Il valore finale risulta, quindi:
 \[f^{\star}_2 = (6.60 \pm 0.01) \cm,\]
 dove è stata considerata la somma tra gli errori statistici e gli errori sistematici su $P_O$, rispetto ai quali gli
 errori statistici precedentemente considerati risultano trascurabili.
 Il valore più giusto tra i due, alla luce dell'$\mathcal{F}$-test effettuato e delle considerazioni sugli errori sistematici, è il secondo:
 infatti la semplicità della retta interpolante a singolo parametro libero permette un più semplice studio degli errori collegati
 al posizionamento dell'oggetto. Un approccio simile per la ricerca dell'errore sistematico nel caso del doppio parametro libero
 renderebbe impossibile l'approssimazione dell'incertezza attraverso le formule sopra citate, rendendo difficoltosa una pratica stima
 della stessa. Il valore ottenuto va comunque corretto tramite i coefficienti di aberrazione sferica, come verrà discusso più avanti.
