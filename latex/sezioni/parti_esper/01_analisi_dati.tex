%**01_tab_1.tx**


I risultati delle misure sono nella \autoref{tab:01_tab_1.tex}.
\begin{tabella}
	\centering
	\begin{center}
\begin{tabulary}{\textwidth}{CCC}
\toprule
Posizione relativa della lente & Diametro fascio a $30 \cm$ & Diametro fascio a $130 \cm$ \\ \midrule
0.400 & 1.187 & 1.000 \\ \midrule
0.450 & 1.193 & 1.105 \\ \midrule
0.460 & 1.171 & 1.172 \\ \midrule
0.475 & 1.168 & 1.190 \\ \midrule
0.500 & 1.186 & 1.254 \\ \midrule
0.550 & 1.190 & 1.367 \\
\bottomrule
\end{tabulary}
\end{center}

	\caption{Risultati autocollimazione $[\cm\,]$}
	\label{tab:01_tab_1.tex}
\end{tabella}
Per la stima della posizione del fuoco e il calcolo dell'incertezza, si considera la seguente formula:
\begin{equation} \label{eq:autocollimazione}
f = P_L - P_O + \left(\mu _0 - \mu ^{\star}\right) + \frac{\mathrm{d}r}{2} - \frac{\overline{PP'}}{2}.
\end{equation}
Per la stima di $ \mu ^{\star} $, valore del micrometro per cui il fascio \`e parallelo, si sono interpolati i diametri del fascio in ordinata con i valori indicati dal micrometro: $ \mu ^{\star}$ \`e individuato dall'ascissa dell'intersezione delle due rette interpolanti ($y=ax+b$ per la prima, con notazione simile per la seconda) nel \autoref{fig:01_graph_1.tex}, i cui parametri sono nella \autoref{tab:01tab2}.
\begin{tabella}
	\centering
	\begin{tabulary}{\textwidth}{CCCCC}
\toprule
Da vicino & $a$ & $\sigma_a$ & $b [\cm]$ & $\sigma_b [\cm]$ \\ \cmidrule{2-5}
 & 11.7 & 0.5 & 0.2 & 1.0\\ \midrule
Da lontano & $a'$ & $\sigma_{a'}$ & $b' [\cm]$ & $\sigma_{b'} [\cm]$ \\ \cmidrule{2-5}
 & 0.1 & 0.7 & 25 & 1  \\
\bottomrule
\end{tabulary}

	\caption{Coefficienti delle rette interpolanti}
	\label{tab:01tab2}
\end{tabella}

\begin{grafico} \centering \input{../gnuplot/immagini/01_graph_1.tex} \caption{Interpolazione lineare} \label{fig:01_graph_1.tex} \end{grafico}

\[ \mu ^{\star} = x_{\mathrm{intersezione}} = \frac{b - b'}{a' - a}  = 4.73 \mm , \]
da cui, dall'equazione \eqref{eq:autocollimazione} grazie alla formula di propagazione quadratica, si ottiene, considerando $\left(a, b\right)$, $\left(a', b'\right)$ rispettivamente correlati e le rette tra loro indipendenti,
% split va dentro a un'equazione
\begin{equation*}
\begin{split} % align numera tutte le righe e se devo toglierne una devo usare \nonumber alla fine della riga, align* nessuna. Meglio split che numera solo una volta
	\sigma^2_{\mu ^{\star}\left(a, a', b, b'\right)}  &= \left(\left.\frac{\partial F}{\partial a}\right|_{x_{\mathrm{int}}}\right)^2   \sigma^2_{a} + \left(\left.\frac{\partial F}{\partial a'}\right|_{x_{\mathrm{int}}}\right)^2   \sigma^2_{a'} + \left(\left.\frac{\partial F}{\partial b}\right|_{x_{\mathrm{int}}}\right)^2   \sigma^2_{b} +\\
								&+ \left(\left.\frac{\partial F}{\partial b'}\right|_{x_{\mathrm{int}}}\right)^2   \sigma^2_{b'} + 2\left(\left.\frac{\partial F}{\partial a}\right|_{x_{\mathrm{int}}}\right)\left(\left.\frac{\partial F}{\partial b}\right|_{x_{\mathrm{int}}}\right)   \cov\left(a, b\right) + \\
								&+ 2\left(\left.\frac{\partial F}{\partial a'}\right|_{x_{\mathrm{int}}}\right)\left(\left.\frac{\partial F}{\partial b'}\right|_{x_{\mathrm{int}}}\right)^2   \cov \left(a', b'\right)
\end{split}
\end{equation*}
che, sotto radice quadrata, d\`a l'incertezza per $ \mu ^{\star} $, considerandolo distribuito normalmente.
Svolgendo i calcoli, si trova
\begin{equation}
\begin{split}
\sigma^2_{\mu ^{\star}}  &=  \left(\left.\frac{b - b'}{\left(a' - a\right)^2}\right|_{x_{\mathrm{int}}}\right)^2 \left(\sigma^2_{a} + \sigma^2_{a'}\right) + \left(\left.\frac{1}{a'- a}\right|_{x_{\mathrm{int}}}\right)^2   \left(\sigma^2_{b} + \sigma^2_{b'}\right) \\
						 &+ 2 \left(\left.\frac{1}{a'- a}\right|_{x_{\mathrm{int}}}\right) \left(\left.\frac{b - b'}{\left(a' - a\right)^2}\right|_{x_{\mathrm{int}}}\right) \big(\cov\left(a, b\right) + \cov(a', b') \big).
\end{split}
\end{equation}
Calcoliamo le covarianze:
\[ \cov\left(a, b\right) = -\frac{\sum_{i} x_i}{\Delta}\sigma_y^2 \] 
dove $\Delta$ \`e il parametro di interpolazione lineare\footnote{M. Loreti, \textit{Teoria degli Errori e Fondamenti di Statistica}, p. 266}; vale lo stesso per a' e b', con le adeguate (x, y).
\[\cov(a, b) = -0.501 \cm^2 \] 
\[\cov(a', b') = -0.895 \cm^2. \]
L'incertezza cos\`i calcolata risulta di $0.0003\cm$, molto bassa a causa della precisione micrometrica, migliorata grazie al fit lineare. Tuttavia si ritiene pi\`u corretto considerarla non pi\`u bassa dell'incertezza strumentale. L'intersezione \`e quindi stimata come
\[ \mu ^{\star} =  \left(0.473 \pm 0.001\right) \cm. \] 

Per rendere visivamente apprezzabile l'incertezza sulle rette, \`e stato creato un grafico (\autoref{fig:01_graph_2.tex}) contenente le rette tracciate per i valori estremali della quota e del coefficiente angolare.
\begin{grafico} \centering \input{../gnuplot/immagini/01_graph_2.tex} \caption{Incertezza sulle rette} \label{fig:01_graph_2.tex} \end{grafico}
%
Si trova per il fuoco il~valore 
\[ f^{\star}_1=\left(6.72 \pm 0.07\right) \cm , \] 
si \`e considerata trascurabile l'incertezza su $\mu^{\star}$ (la scelta di considerare l'incertezza strumentale non influisce quindi sugli altri risultati) rispetto a quella fornita dal laboratorio su $P_L$ e $P_O$, pari a $\sigma _P = 0.05 \cm$: ci\`o porta propagando a un'incertezza di $\sqrt{2}   \sigma_P = 0.07 \cm$. Tale stima va considerata a meno della correzione di aberrazione sferica (per la quale cfr. \autoref{subsec:aberrazione_sferica}).

