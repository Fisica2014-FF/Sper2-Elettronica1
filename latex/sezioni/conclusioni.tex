Le grandezze fisiche della lente cercate sono presentate nella seguente lista:
\begin{itemize}
\item $f_{D,1} = (6.76 \pm 0.08) \cm$
\item $f_{D,2} = (6.65 \pm 0.05) \cm$
\item $f_{D,3} = (6.63 \pm 0.05) \cm$
\item $f_D = (6.66 \pm 0.03) \cm$
\item $c = (1.7 \pm 0.1)$
\item $t = (0.194 \pm 0.002) \cm$
\item $A = (0.113 \pm 0.005) \cm$
\item $V = (56 \pm 1)$  
\end{itemize}
(dove il fuoco \`e inteso nel giallo).

Per il calcolo delle grandezze fisiche non dirette (per esempio per il calcolo della media del fuoco o per il calcolo della costante di aberrazione sferica $c$) si \`e operato con i valori ottenuti nell'analisi dati non approssimati, l'approssimazione \`e stata fatta solo in fase di presentazione dei dati stessi.

Le stime dei fuochi sono sufficientemente compatibili tra loro e con la deviazione massima della prima stima è giustificabile con la maggiore incertezza legata alla metodologia di misura; la maggior parte dell'incertezza della seconda e terza stima \`e invece xdovuta alla correzione per aberrazione sferica (vedi \autoref{subsec:aberrazione_sferica}).

Le stime dei coefficienti di aberrazione sono in linea con le aspettative teoriche.
