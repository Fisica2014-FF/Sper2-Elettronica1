\documentclass[11pt,a4paper]{article} % Prepara un documento con un font grande

\usepackage{iftex}

\ifLuaTeX
  \input{./preamboli_e_stili/pacchetti_luatex.tex}
\else
  \input{./preamboli_e_stili/pacchetti.tex}
\fi
\DeclareGraphicsExtensions{.pdf, .png, .jpg} % Se due immagini hanno lo stesso nome sceglile secondo l'ordine di filetype qui
\graphicspath{ {./img/} }					 % Path delle immagini 

\title{
\vspace{-2cm}
\fontsize{36pt}{10pt}\selectfont
RELAZIONE DI \\[8mm] ELETTRONICA \\[12mm]
Bozza di analisi dati
}
% \title{}
\author{
\large
\textsc{Francesco Forcher}\\[2mm]
\normalsize Università di Padova, Facoltà di Fisica\\
\normalsize francesco.forcher@studenti.unipd.it\\
\normalsize Matricola: \texttt{1073458}\\
\and
\large
\textsc{Davide Chiappara}\\[2mm]
\normalsize Università di Padova, Facoltà di Fisica\\
\normalsize davide.chiappara@studenti.unipd.it\\
\normalsize Matricola: \texttt{1070160}\\
\and
\large
\textsc{Gabriele Labanca}\\[2mm]
\normalsize Università di Padova, Facoltà di Fisica\\
\normalsize gabriele.labanca@studenti.unipd.it\\
\normalsize Matricola: \texttt{1069556}
}
\date{\today}

\input{./preamboli_e_stili/stili_float.tex}


%////////////////////////////////////////////////////////////////////////////////////////////////////////////////////////////
%////////////////////////////////////////////////////////////////////////////////////////////////////////////////////////////
% Fine dei dati iniziali per il latex: il documento finale inizierà da qui
\begin{document}
{
\maketitle % Produce il titolo a partire dai comandi \title, \author e \date


% REGEX per modificare col textrm _\{([a-zA-Z0-9\,]{2,})\}   _{textrm{\1}}

\section{Misure dirette di resistenze}
I valori riportati in tabella (valori in $\Omega$) sono quelli delle misure dirette delle resistenze, prese col multimetro FLUKE 111; il fondo scala \`e di $200mA$ per le correnti e di $600mV$ per le tensioni.

%tabella RES DIRETTE
\begin{tabella}
	\centering
	\input{./tabelle/01_tab_1.tex}
	\caption{Misure dirette resistenze}
	\label{tab:01_tab_1.tex}
\end{tabella}


Per stimare gli errori si \`e usata la formula seguente:
%formula diretta
\[ \sigma_R =0.58 \sqrt{ \sigma_{\textrm{sist}} + \sigma_{\textrm{stat}}}= 0.58 \sqrt{(R\cdot\Delta P + n_{\textrm{digit}} \cdot \min(\textrm{FS}))}\]
Infatti gli errori legati alla misurazione sono dovuti sia a errori di scala ($ R= k _R \cdot R^{(r)} $), sia a errori casuali connessi al numero di digit. Per chiarezza di notazione, $\sigma^{(r)}$ \`e considerato errore statistico, mentre con $\sigma$ si intende l'errore totale.

Per quanto riguarda le resistenze $R_5$ e $R_6$ in serie, da una misurazione diretta effettuata col multimetro FLUKE 111 risulta che $R_{\textrm{S,sper}}= (402 \pm 4 )\Omega$. Col calcolo teorico, il valore di tale resistenza equivalente risulta invece $R_{\textrm{S, teor}}=(402 \pm 2 )\Omega$, dove per l'errore teorico \`e stato considerato che $R_{\textrm{S,teor}}=k\cdot(R_1^{(r)} + R_2^{(r)})$, infatti la k \`e costante in misurazioni successive, mantenendo il medesimo fondo scala. Con semplice propagazione degli errori risulta che 
\[\sigma_{R_{\textrm{S,teor}}}=\sqrt{(R_1+R_2)^2\cdot\sigma_k^2+\sigma_{R_1^{(r)}}^2+\sigma_{R_2^{(r)}}^2}\]
dove $\sigma_k$ \`e stata ricavata dall'errore percentuale fornito dal costruttore del multimetro e considerando k distribuito uniformemente:
\(\sigma_k=0.58 \cdot Err\%\). 
\`E stata calcolata la correlazione tra le due diverse stime della resistenza, considerando che la loro differenza dovrebbe essere nulla:
\[\lambda=\frac{\left|(R_{\textrm{S,teor}}-R_{\textrm{S,sper}})-0\right|}{\sigma_{R_{\textrm{S,teor}}-R_{\textrm{S,sper}}}}=0.3\]
con $R_{\textrm{S,teor}}-R{S,sper}=k\cdot(R_{\textrm{S,sper}}^{(r)}-R_1^{(r)}-R_2^{(r)})$, da cui per propagazione si ricava che 
\[ \sigma_{R_{\textrm{S,teor}}} - \sigma{R_{\textrm{S,sper}}} = 
        \sqrt{ (R_{\textrm{S,teor}} -R_{\textrm{S,sper}})^2 \sigma_k^2 + 3 \sigma_R^{(r) 2} }\]

Per il calcolo della resistenza equivalente a $R_5$ e $R_6$ in parallelo, il valore misurato con il multimetro FLUKE 111 \`e $R_{\textrm{P,sper}}= (94 \pm 4) \Omega$.
Il valore teorico \`e $R_{\textrm{P,teor}}=(94 \pm 0.5) \Omega$: considerando che $R_{\textrm{P,teor}}=k  \frac{R_5^{(r)} R_6^{(r)}}{R_5^{(r)} + R_6^{(r)}}$ e propagando, riutilizzando la medesima semplificazione sull'errore di scala, si ottiene 
\[\sigma_{\textrm{P,teor}}=\sqrt{\left(\frac{R_5 R_6}{R_5+R_6}\right)^2 \sigma_k^2 + \frac{R_5^4 + R_6^4}{(R_5 + R_6)^4}  \sigma_{R_{\textrm{P,teor}}^{(r) 2}}} .\]
Per il calcolo della compatibilit\`a, si sono utilizzate le medesime formule che per le resistenze in serie, opportunamente adattate:
\[\lambda=
	\frac{ \left|(R_{\textrm{P,teor}}-R_{\textrm{P,sper}})-0 \right| } { \sigma_{R_{\textrm{P,teor}}}-\sigma_{R_{\textrm{P,sper}}}}=0.14.\] 

Nota bene: tutti i calcoli sono stati effettuati mantenendo un numero superiore di cifre significative, riducendone il numero solo in sede di presentazione dati.



 











\section{Misura voltamperometrica di una resistenza}
Per misurare una resistenza piccola \`e stato costruito un circuito come in figura (da aggiungersi). Una prima misura diretta \`e stata effettuata utilizzando il multimetro FLUKE 111, che \`e risultata $R_x=(3.0 \pm 0.1) \Omega$.
Costruito il circuito, si \`e variata la resistenza di carico e la potenza erogata dal generatore per indagare di quanto fosse la caduta di potenziale al variare della corrente che attraversa R. I dati ottenuti sono riportati in tabella. 

%tabella AMPERE-VOLT
\begin{tabella}
	\centering
	\input{./tabelle/02_tab_1.tex}
	\caption{Misure caduta di potenziale}
	\label{tab:02_tab_1.tex}
\end{tabella}

In grafico sono riportate tali misure esprimendo V in funzione di I, sovrapposte a un fit lineare ottenuto col metodo della massima verosimiglianza.

\begin{grafico}
\centering
\includegraphics[width=\textwidth]{02_fitlin}
\caption{Fit lineare}
\label{fig:fitlin}
\end{grafico}

I coefficienti della retta interpolante $y=mx+c$ sono:
\[m = (2.809 \pm 0.004) \Omega \]
\[c = (0.2 \pm 0.2) mV.\]
%Calcolando la covarianza tra $m$ e $c$ si ottiene 
%\[cov(m, c) = -0.000660084 V^4\] %UDM
Si \`e calcolata la correlazione 
\[\rho(m, c) = \frac{cov(m, c)}{\sigma_m \sigma_c}=-0.11\]
e l'errore a posteriori sulla caduta di tensione \`e di $\sigma_V=0.6V$.

A seguire il grafico dei residui: si \`e rappresentata la differenza tra il valore di tensione misurato e quello ricavato teoricamente dalla retta interpolante in corrispondenza del suo valore di corrente.

\begin{grafico}
\centering
\includegraphics[width=\textwidth]{02_residui}
\caption{Residui}
\label{fig:residui}
\end{grafico}

Una stima della resistenza \`e data dalla pendenza della retta interpolante. Tale retta ha un errore che \`e composizione di un errore sistematico e di uno statistico, infatti si pu\`o scrivere $m=\frac{k_V (V_2^{(r)}-V_1^{(r)})}{k_i (i_2^{(r)} - i_1^{(r)})}=\frac{k_V}{k_i}m^{(r)}$.
Da una propagazione risulta che l'errore su tale grandezza \`e $\sigma_m=\sqrt{\sigma_{\textrm{m,fit}}^2 + \sigma_{k_V}^2 m^2 + \sigma_{k_i}^2 m^2}$ con $\sigma_{\textrm{m,fit}}$ errore casuale ottenuto dall'interpolazione.
Risulta che l'incertezza sulla resistenza \`e quasi completamente data dall'errore sistematico. Il risultato finale \`e $R=(2.8 \pm 0.4) \Omega$; l'errore percentuale \`e del $13 \%$.

Si possono confrontare il risultato teorico e quello sperimentale con un calcolo di compatibilit\`a. Dato che sono state usate strumentazioni differenti per le due stime, se ne pu\`o applicare la definizione: 
$\lambda=\frac{|R_x - R|}{\sqrt{\sigma_{R_x}^2+\sigma_R^2}}=0.5$.
 











\section{Resistenze interne degli strumenti di misura}
Attraverso costruzioni di circuiti o misure dirette, si sono stimate le resistenze interne degli strumenti utilizzati.
Per la stima della resistenza interna del generatore si \`e costruito un circuito come in figura (da aggiungersi) e utilizzato il voltmetro AGILENT U1232A con l'amperometro BECKMAN T110B.
Dalle misure risulta che
\begin{align}
V_0 &=(5.01 \pm 0.01 )V \ \textrm{con}\  V_{\textrm{FS}}=6V \\
i   &=(124.9 \pm 0.5) mA \ \textrm{con}\  i_{\textrm{FS}}=200mA \\
V   &=(5.00 \pm 0.01) V \ \textrm{con}\  V_{\textrm{FS}}=6V.
\end{align}
Da uno studio del circuito si ricava la formula $R_G=\frac{V_0-V}{V}$.
Stimandone l'errore, per evitare problemi di correlazione si pu\`o scrivere $R_G=\frac{k_v (V_0^{(r)}- V^{(r)})}{i}$, da cui propagando: 
\[\sigma_{R_G}=\sqrt{R_G^2 \sigma_{k_V}^2 + \frac{(\sigma_{V^{(r)}}^2 + \sigma_{V_0^{(r)}}^2)}{i^2} + \frac{(V_0-V)^2}{i^4} \sigma_i^2},\] ricordando che per $\sigma_i$ si intende l'errore strumentale totale. Concludendo, $R_G= (0.10 \pm 0.16) \Omega$.

Un diverso circuito \`e stato costruito per stimare la resistenza interna dell'AGILENT U1232A utilizzato come voltmetro.
Una misurazione diretta di $R_V$ \`e stata ottenuta utilizzando come ohmetro il BECKMAN T110B: $R_{\textrm{V, sper}}=(11.2 \pm 0.1) M\Omega$, con fondo scala di $20 M\Omega$. 
Le misure prese a circuito chiuso sono: 
\begin{align}
R_S &= (0.990 \pm 0.005) M\Omega \ \textrm{con}\  R_{\textrm{FS}}=6 M\Omega \\
V_0 &= (5.01 \pm 0.01) V \ \textrm{con}\  V_{\textrm{FS}} = 6 V \\
V &= (4.60 \pm 0.01) V \ \textrm{con}\  V_{\textrm{FS}} = 6 V
\end{align}
Studiando il circuito, si pu\`o dimostrare che 
\[ R_{\textrm{V, teor}} = \frac{R_S V}{V_0 - V} . \]
Portando fuori dai valori il coefficiente $k_V$ e semplificandolo, si ha $R_{\textrm{V, teor}} = \frac{R_S V^{(r)}}{V_0^{(r)} - V^{(r)}}$ da cui, propagando, si ottiene \[\sigma_{R_{\textrm{V,teor}}} = \sqrt{\sigma_{R_S}^2 \left(\frac{V}{(V_0 - V)^2} \right)^2 + \sigma_{V_{(r)}}^2 \left(\frac{R_S V_0}{(V_0 - V)^2}\right)^2 + \sigma_{V_0^{(r) 2}} \left(\frac{R_S V}{(V_0 - V)^2}\right)^2}.\]
Risulta $R_{\textrm{V, teor}} = (11.19 \pm 0.08) M\Omega$.


Per misurare la resistenza interna del BECKMAN T110B, usato come amperometro, si \`e semplicemente effettuato un collegamento con il FLUKE 111 usato come ohmetro. I valori sono riportati in tabella.

%tabella RESISTENZE AMPEROMETRO
\begin{tabella}
	\centering
	\input{./tabelle/03_tab_1.tex}
	\caption{Resistenze dell'amperometro BECKMAN}
	\label{tab:03_tab_1.tex}
\end{tabella}


























\end{document}
