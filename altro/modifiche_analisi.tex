 % sotto allo SigmaV=0.6 (milli?)Volt
 Questo valore è stato calcolato direttamente dai dati con la formula\footnote{1
 M. Loreti, \itshape{Teoria degli Errori e Fondamenti di Statistica}, p. 666
}
\begin{equation}
 \sigma_V = \frac{\sum_i^N \left(y_i-(m x_i+q)^2 \right)}{N-2}.
\end{equation}
Risulta un po' alto, ma si vede anche dal grafico \autoref{fig:residui} come un
valore minore di $\sigma_V$ implicherebbe che la maggior parte dei dati sarebbe a più
di una sigma dallo zero.

\S\ref{sec:appendice}


file:///home/francesco/Dropbox/Laboratorio/Elettronica_1/Sper2-Elettronica1/altro/Eletr1_ResAmperometro.pdf
file:///home/francesco/Dropbox/Laboratorio/Elettronica_1/Sper2-Elettronica1/altro/Eletr1_ResGeneratore.pdf
file:///home/francesco/Dropbox/Laboratorio/Elettronica_1/Sper2-Elettronica1/altro/Eletr1_ResGeneratore2.pdf
file:///home/francesco/Dropbox/Laboratorio/Elettronica_1/Sper2-Elettronica1/altro/Eletr1_respar.pdf
file:///home/francesco/Dropbox/Laboratorio/Elettronica_1/Sper2-Elettronica1/altro/Eletr1_ResSerie.pdf
file:///home/francesco/Dropbox/Laboratorio/Elettronica_1/Sper2-Elettronica1/altro/Eletr1_ResVoltmetro.pdf
file:///home/francesco/Dropbox/Laboratorio/Elettronica_1/Sper2-Elettronica1/altro/Eletr1_VoltAmpere.pdf
\begin{figure}[0.7\textwidth]
 \centering 
 \input{02_graph_4.tex} 
 \caption{Eletr1_ResSerie.pdf} 
 \label{gr:02_graph_4.tex}
\end{figure}

\begin{figure}[0.7\textwidth]
 \centering 
 \input{02_graph_4.tex} 
 \caption{Eletr1_respar.pdf} 
 \label{gr:02_graph_4.tex}
\end{figure}

\begin{figure}[0.7\textwidth]
 \centering 
 \input{02_graph_4.tex} 
 \caption{Eletr1_VoltAmpere.pdf} 
 \label{gr:02_graph_4.tex}
\end{figure}

\begin{figure}[0.7\textwidth]
 \centering 
 \input{02_graph_4.tex} 
 \caption{Eletr1_ResSerie.pdf} 
 \label{gr:02_graph_4.tex}
\end{figure}\begin{figure}[0.7\textwidth]
 \centering 
 \input{02_graph_4.tex} 
 \caption{Eletr1_ResGeneratore.pdf} 
 \label{gr:02_graph_4.tex}
\end{figure}

\begin{figure}[0.7\textwidth]
 \centering 
 \input{02_graph_4.tex} 
 \caption{Eletr1_ResGeneratore2.pdf} 
 \label{gr:02_graph_4.tex}
\end{figure}

\begin{figure}[0.7\textwidth]
 \centering 
 \input{02_graph_4.tex} 
 \caption{Eletr1_ResVoltmetro.pdf} 
 \label{gr:02_graph_4.tex}
\end{figure}

\begin{figure}[0.7\textwidth]
 \centering 
 \input{02_graph_4.tex} 
 \caption{Eletr1_ResAmperometro.pdf} 
 \label{gr:02_graph_4.tex}
\end{figure}